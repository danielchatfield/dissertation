\documentclass[final,dissertation.tex]{subfiles}

%TC:ignore

\begin{document}
  \chapter*{Proforma}

  {\large
  \begin{table}[h]
  \begin{tabularx}{\textwidth}{ll}
  Name:               & \bf Daniel Chatfield                      \\
  College:            & \bf Robinson College                      \\
  Project Title:      & \bf Security Countermeasures for \mifare{} Classic RFID cards \\
  Examination:        & \bf Computer Science Tripos --- Part II, 2016  \\
  Word Count:         & \bf \SI{11995}{}\footnote{}  \\
  Project Originator: & Dr Markus Kuhn                    \\
  Supervisor:         & Dr Markus Kuhn                    \\
  \end{tabularx}
  \end{table}
  }

  \footnotetext{Computed using \texttt{texcount -total -template={SUM} *.tex}}

  \section*{Original Aim of the Project}

  The original aim of the project, as set out in the proposal, was to develop countermeasures that mitigated the risk of attack against \mifare{} Classic cards. The proposed countermeasures consisted of a digital signature library and a gossip protocol for distributing revoked UIDs.

  I also planned to write a \mifare{} library which would be used by the countermeasures to communicate with \mifare{} Classic cards.

  \section*{Work Completed}

  The project encompasses the original aims. I have produced a \mifare{} library, digital signature library and a revocation gossip protocol.

  In addition to my original aims, I have also devised several additional countermeasures and a comprehensive simulation suite for simulating networks of readers and cards.

  \section*{Special Difficulties}
  None
\end{document}
%TC:endignore
